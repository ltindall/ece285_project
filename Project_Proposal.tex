\documentclass{article} % For LaTeX2e
\usepackage{nips15submit_e,times}
\usepackage{hyperref}
\usepackage{url}
\usepackage{graphicx}
\usepackage{amsmath}
\renewcommand{\arraystretch}{1.5}
\raggedbottom
\usepackage{booktabs}       % professional-quality tables
\usepackage{listings}
\usepackage{multirow}
\usepackage{subfigure}

%\documentstyle[nips14submit_09,times,art10]{article} % For LaTeX 2.09


\title{Our Sick Computer Vision Project}

\author{
Phat ~Phan\\
\texttt{p6phan@eng.ucsd.edu} \\ \And
Christian ~Koguchi\\
\texttt{ckoguchi@eng.ucsd.edu} \\ \And
Lucas ~Tindall\\
\texttt{ltindall@eng.ucsd.edu} \\ 
}


\newcommand{\fix}{\marginpar{FIX}}
\newcommand{\new}{\marginpar{NEW}}

\nipsfinalcopy % Uncomment for camera-ready version

\begin{document}

\maketitle

%%%%%%%%%%%%%%%%%%%%%%%%%%%%%%%%%%%%%%%%%%%%%%
% Abstract
%%%%%%%%%%%%%%%%%%%%%%%%%%%%%%%%%%%%%%%%%%%%%%
\begin{abstract}

\end{abstract}

%%%%%%%%%%%%%%%%%%%%%%%%%%%%%%%%%%%%%%%%%%%%%%
% Objectives
%%%%%%%%%%%%%%%%%%%%%%%%%%%%%%%%%%%%%%%%%%%%%%
\section{Objectives}
Our project will focus on using Generative Adversarial Networks (GANs) for multiple image transformation tasks. Our first objective is to use GANs for image colorization. Next, we will implement artistic style transfer. Lastly, we will perform season transfer. For each of these tasks, we will focus on natural images of landscapes and cityscapes. We will source our artistic styles from collections of paintings including the works of Vincent Van Gogh and Claude Monet. 


%%%%%%%%%%%%%%%%%%%%%%%%%%%%%%%%%%%%%%%%%%%%%%

%%%%%%%%%%%%%%%%%%%%%%%%%%%%%%%%%%%%%%%%%%%%%%
% Challenges
%%%%%%%%%%%%%%%%%%%%%%%%%%%%%%%%%%%%%%%%%%%%%%
\section{Challenges}

GANs are difficult to train.  An imbalance in performance between the Generator network and the Discriminator network could lead to poor performance and instablity since an overpowering Discriminator will cause the Generator to be unable to learn and an overpowering Generator will be unable to improve from a weak Discriminator. 

It is difficult to quantitatively test the performance of GANs using an evaluation metric without being subjective or relying on heuristics specific to the data.  It is hard to compare performance across different domains and datasets in a systematic way.  



%%%%%%%%%%%%%%%%%%%%%%%%%%%%%%%%%%%%%%%%%%%%%%

%%%%%%%%%%%%%%%%%%%%%%%%%%%%%%%%%%%%%%%%%%%%%%
% Dataset
%%%%%%%%%%%%%%%%%%%%%%%%%%%%%%%%%%%%%%%%%%%%%%
\section{Dataset}


%%%%%%%%%%%%%%%%%%%%%%%%%%%%%%%%%%%%%%%%%%%%%%



%%%%%%%%%%%%%%%%%%%%%%%%%%%%%%%%%%%%%%%%%%%%%%
% References
%%%%%%%%%%%%%%%%%%%%%%%%%%%%%%%%%%%%%%%%%%%%%%
\section{References}

Precomputed Real-Time Texture Synthesis with Markovian Generative Adversarial Networks: \\ 
\quad https://arxiv.org/pdf/1604.04382.pdf \\ 

Unsupervised Cross-Domain Image Generation: \\ 
\quad https://arxiv.org/pdf/1611.02200.pdf \\ 

Image-to-Image Translation with Conditional Adversarial Networks: \\ 
\quad https://arxiv.org/pdf/1611.07004.pdf \\ 

GAN Hacks:  \\ 
\quad https://github.com/soumith/ganhacks \\ 

Towards Principled Methods For Training Generative Adversarial Networks: \\ 
\quad https://openreview.net/pdf?id=Hk4\_qw5xe \\ 

Improved Techniques for Training GANs: \\ 
\quad https://arxiv.org/pdf/1606.03498.pdf \\ 

%%%%%%%%%%%%%%%%%%%%%%%%%%%%%%%%%%%%%%%%%%%%%%












\end{document}
